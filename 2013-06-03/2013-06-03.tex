\documentclass[aspectratio=169]{beamer}

% Suppress the navigation bar
\beamertemplatenavigationsymbolsempty

% Change the font
\usepackage{fontspec}
\setmainfont{DejaVu Serif}
\setsansfont{DejaVu Sans}
\setmonofont{DejaVu Sans Mono}
\setbeamerfont{title}{series=\bfseries}
\setbeamerfont{frametitle}{series=\bfseries}

% Change the themes
\usecolortheme[accent=blue, dark]{solarized}
\useinnertheme{circles}
\setbeamercolor{item projected}{bg=solarizedRebase0, fg=solarizedRebase03}
\setbeamercolor{solarizedcolorbox}{bg=solarizedRebase02, fg=solarizedRebase0}

% Custom packages
\usepackage{listings}
\lstset{basicstyle=\ttfamily}
\usepackage{tikz}
\usetikzlibrary{positioning}
\newenvironment{tikztree}[1][3em]{\begin{center}
\begin{tikzpicture}
  \tikzstyle{every node}=[draw=solarizedAccent, circle,
                           minimum size=#1]}{\end{tikzpicture}
\end{center}}

% Custom title page
\defbeamertemplate*{title page}{customized}[1][]
{ 
  \begin{flushright}
    {\usebeamerfont{institute}\insertinstitute}
  \end{flushright}
  \begin{beamercolorbox}[wd=\paperwidth, sep=2em]{solarizedcolorbox}
    {\usebeamerfont{title}\usebeamercolor[fg]{title}\inserttitle}
  \end{beamercolorbox}
  \begin{flushright}
    {\usebeamerfont{author}\insertauthor}

    {\usebeamerfont{institute}\insertdate}
  \end{flushright}
}
\title{Parsers}
\author{Jon Eyolfson}
\date{June 3, 2013}
\institute{University of Waterloo}

\begin{document}
%%%%%%%%%%%%%%%%%%%%%%%%%%%%%%%%%%%%%%%%%%%%%%%%%%%%%%%%%%%%%%%%%%%%%%%%%%%%%%%%
\begin{frame}[plain, noframenumbering]
  \titlepage
\end{frame}
%%%%%%%%%%%%%%%%%%%%%%%%%%%%%%%%%%%%%%%%%%%%%%%%%%%%%%%%%%%%%%%%%%%%%%%%%%%%%%%%

%%%%%%%%%%%%%%%%%%%%%%%%%%%%%%%%%%%%%%%%%%%%%%%%%%%%%%%%%%%%%%%%%%%%%%%%%%%%%%%%
\begin{frame}
\frametitle{Goals}

Formulate a grammar for integer arithmetic\\~\\

Create an AST and interpreter for our grammar
\end{frame}
%%%%%%%%%%%%%%%%%%%%%%%%%%%%%%%%%%%%%%%%%%%%%%%%%%%%%%%%%%%%%%%%%%%%%%%%%%%%%%%%

%%%%%%%%%%%%%%%%%%%%%%%%%%%%%%%%%%%%%%%%%%%%%%%%%%%%%%%%%%%%%%%%%%%%%%%%%%%%%%%%
\begin{frame}
\frametitle{Our Grammar}

Here is our attempt at the grammar without multiplication and division for now:
\\~\\

\begin{align*}
T &= \{ +, -, \text{\lstinline{INT}} \}\\
N &= \{ e \}\\
S &= e\\
P &= \begin{cases}e \rightarrow e + e \\
     e \rightarrow e - e\\
     e \rightarrow \text{\lstinline{INT}}\end{cases}
\end{align*}
\end{frame}
%%%%%%%%%%%%%%%%%%%%%%%%%%%%%%%%%%%%%%%%%%%%%%%%%%%%%%%%%%%%%%%%%%%%%%%%%%%%%%%%

%%%%%%%%%%%%%%%%%%%%%%%%%%%%%%%%%%%%%%%%%%%%%%%%%%%%%%%%%%%%%%%%%%%%%%%%%%%%%%%%
\begin{frame}[plain, noframenumbering]
  \begin{beamercolorbox}[wd=\paperwidth, sep=2em]{solarizedcolorbox}
    {\usebeamerfont{title}\usebeamercolor[fg]{title} \hfill Start AST}
  \end{beamercolorbox}
\end{frame}
%%%%%%%%%%%%%%%%%%%%%%%%%%%%%%%%%%%%%%%%%%%%%%%%%%%%%%%%%%%%%%%%%%%%%%%%%%%%%%%%

%%%%%%%%%%%%%%%%%%%%%%%%%%%%%%%%%%%%%%%%%%%%%%%%%%%%%%%%%%%%%%%%%%%%%%%%%%%%%%%%
\begin{frame}
\frametitle{Abstract Syntax Trees}

An AST is a tree representation of the input language\\~\\

This tree should only represents the important details of the input\\~\\
\end{frame}
%%%%%%%%%%%%%%%%%%%%%%%%%%%%%%%%%%%%%%%%%%%%%%%%%%%%%%%%%%%%%%%%%%%%%%%%%%%%%%%%

%%%%%%%%%%%%%%%%%%%%%%%%%%%%%%%%%%%%%%%%%%%%%%%%%%%%%%%%%%%%%%%%%%%%%%%%%%%%%%%%
\begin{frame}
\frametitle{Unimportant Details in Our Grammar}

Our parse trees have the following unimportant details:
\begin{itemize}
  \item Rule names
  \item Brackets (or grouping)\\~\\
\end{itemize}
\end{frame}
%%%%%%%%%%%%%%%%%%%%%%%%%%%%%%%%%%%%%%%%%%%%%%%%%%%%%%%%%%%%%%%%%%%%%%%%%%%%%%%%

%%%%%%%%%%%%%%%%%%%%%%%%%%%%%%%%%%%%%%%%%%%%%%%%%%%%%%%%%%%%%%%%%%%%%%%%%%%%%%%%
\begin{frame}
\frametitle{Our AST Nodes}

\begin{description}
  \item[\textbf{BinaryExpr}] need to know what operator to use, the left and
                             right side
  \item[\textbf{IntegerExpr}] to store the value of an integer\\~\\
\end{description}
Both of which are \textbf{Expr} (which is an abstract base class)
\end{frame}
%%%%%%%%%%%%%%%%%%%%%%%%%%%%%%%%%%%%%%%%%%%%%%%%%%%%%%%%%%%%%%%%%%%%%%%%%%%%%%%%

%%%%%%%%%%%%%%%%%%%%%%%%%%%%%%%%%%%%%%%%%%%%%%%%%%%%%%%%%%%%%%%%%%%%%%%%%%%%%%%%
\begin{frame}
\frametitle{Our AST Node Representations}

We will represent our AST nodes as follows:\\~\\

\begin{columns}[onlytextwidth]
\column{0.5\textwidth}

\centering
\textbf{BinaryExpr}

\begin{tikztree}[3.5em]
  \node {\lstinline{op}}
    child { node {\lstinline[basicstyle=\footnotesize\ttfamily]{left}} }
    child { node {\lstinline[basicstyle=\footnotesize\ttfamily]{right}} };
\end{tikztree}

\lstinline{op} is either $+, -, *$ or $/$\\
\lstinline{left} and \lstinline{right} are \textbf{Exprs}

\column{0.5\textwidth}

\centering
\textbf{IntegerExpr}

\begin{tikztree}[3.5em]
  \node {\lstinline{value}};
\end{tikztree}

\lstinline{value} is an \lstinline{int} (in Java we can pass the regular
expression matched by the \lstinline{INT} token to \lstinline{Integer.parseInt})

\end{columns}
\end{frame}
%%%%%%%%%%%%%%%%%%%%%%%%%%%%%%%%%%%%%%%%%%%%%%%%%%%%%%%%%%%%%%%%%%%%%%%%%%%%%%%%

%%%%%%%%%%%%%%%%%%%%%%%%%%%%%%%%%%%%%%%%%%%%%%%%%%%%%%%%%%%%%%%%%%%%%%%%%%%%%%%%
\begin{frame}
\frametitle{AST Question (1)}

If our input is \lstinline{"1 + 2 + 3"} and our parse tree is the following:

\begin{tikztree}  
  \node {$e$}
    child { node {$e$} 
      child { node {$e$} child { node{\lstinline{INT}} }
      }
      child { node {$+$} }
      child { node {$e$} child { node{\lstinline{INT}} }}
    }
    child { node {$+$} }
    child { node {$e$} child { node{\lstinline{INT}} }};
\end{tikztree}

What is our AST?
\end{frame}
%%%%%%%%%%%%%%%%%%%%%%%%%%%%%%%%%%%%%%%%%%%%%%%%%%%%%%%%%%%%%%%%%%%%%%%%%%%%%%%%

%%%%%%%%%%%%%%%%%%%%%%%%%%%%%%%%%%%%%%%%%%%%%%%%%%%%%%%%%%%%%%%%%%%%%%%%%%%%%%%%
\begin{frame}
\frametitle{AST Answer (1)}

\begin{tikztree}
  \node {$+$}
    child { node {$+$}
        child { node{\lstinline{1}} }
        child { node{\lstinline{2}} }
    }
    child { node {\lstinline{3}} };
\end{tikztree}
\end{frame}
%%%%%%%%%%%%%%%%%%%%%%%%%%%%%%%%%%%%%%%%%%%%%%%%%%%%%%%%%%%%%%%%%%%%%%%%%%%%%%%%

%%%%%%%%%%%%%%%%%%%%%%%%%%%%%%%%%%%%%%%%%%%%%%%%%%%%%%%%%%%%%%%%%%%%%%%%%%%%%%%%
\begin{frame}
\frametitle{AST Question (2)}

Now consider the same input and the following parse tree:

\begin{tikztree}
  \node {$e$}
    child { node {$e$} child { node{\lstinline{INT}} }}
    child { node {$+$} }
    child { node {$e$} 
      child { node {$e$} child { node{\lstinline{INT}} }}
      child { node {$+$} }
      child { node {$e$} child { node{\lstinline{INT}} }}
    };
\end{tikztree}

What is our AST now?
\end{frame}
%%%%%%%%%%%%%%%%%%%%%%%%%%%%%%%%%%%%%%%%%%%%%%%%%%%%%%%%%%%%%%%%%%%%%%%%%%%%%%%%

%%%%%%%%%%%%%%%%%%%%%%%%%%%%%%%%%%%%%%%%%%%%%%%%%%%%%%%%%%%%%%%%%%%%%%%%%%%%%%%%
\begin{frame}
\frametitle{AST Answer (2)}

\begin{tikztree}
  \node {$+$}
    child { node {\lstinline{1}} }
    child { node {$+$}
        child { node{\lstinline{2}} }
        child { node{\lstinline{3}} }
    };
\end{tikztree}
\end{frame}
%%%%%%%%%%%%%%%%%%%%%%%%%%%%%%%%%%%%%%%%%%%%%%%%%%%%%%%%%%%%%%%%%%%%%%%%%%%%%%%%

%%%%%%%%%%%%%%%%%%%%%%%%%%%%%%%%%%%%%%%%%%%%%%%%%%%%%%%%%%%%%%%%%%%%%%%%%%%%%%%%
\begin{frame}[plain, noframenumbering]
  \begin{beamercolorbox}[wd=\paperwidth, sep=2em]{solarizedcolorbox}
    {\usebeamerfont{title}\usebeamercolor[fg]{title} End AST \hfill Start
      Interpreter}
  \end{beamercolorbox}
\end{frame}
%%%%%%%%%%%%%%%%%%%%%%%%%%%%%%%%%%%%%%%%%%%%%%%%%%%%%%%%%%%%%%%%%%%%%%%%%%%%%%%%

%%%%%%%%%%%%%%%%%%%%%%%%%%%%%%%%%%%%%%%%%%%%%%%%%%%%%%%%%%%%%%%%%%%%%%%%%%%%%%%%
\begin{frame}
\frametitle{Our Own Interpreter}

An interpreter executes the AST directly\\~\\

Our result should be equivalent to following all the mathematical rules for
the input\\~\\

The rules in this case is the DMAS part of BEDMAS
\end{frame}
%%%%%%%%%%%%%%%%%%%%%%%%%%%%%%%%%%%%%%%%%%%%%%%%%%%%%%%%%%%%%%%%%%%%%%%%%%%%%%%%

%%%%%%%%%%%%%%%%%%%%%%%%%%%%%%%%%%%%%%%%%%%%%%%%%%%%%%%%%%%%%%%%%%%%%%%%%%%%%%%%
\begin{frame}[fragile]
\frametitle{Our Interpreter Implementation}

\begin{lstlisting}[backgroundcolor=\color{solarizedRebase02}]
public int eval(BinaryExpr e) {
  switch (e.op) {
    case '*':
      return eval(e.left) * eval(e.right);
    case '/':
      return eval(e.left) / eval(e.right);
    case '+':
      return eval(e.left) + eval(e.right);
    case '-':
      return eval(e.left) - eval(e.right);
  }
}
public int eval(IntegerExpr i) {
  return i.value;
}
\end{lstlisting}
\end{frame}
%%%%%%%%%%%%%%%%%%%%%%%%%%%%%%%%%%%%%%%%%%%%%%%%%%%%%%%%%%%%%%%%%%%%%%%%%%%%%%%%

%%%%%%%%%%%%%%%%%%%%%%%%%%%%%%%%%%%%%%%%%%%%%%%%%%%%%%%%%%%%%%%%%%%%%%%%%%%%%%%%
\begin{frame}
\frametitle{AST Evaluation (1)}

We call \lstinline{eval} with the root node, what is the result?

\begin{columns}[onlytextwidth]
\column{0.5\textwidth}
\begin{tikztree}
  \node {$+$}
    child { node {$+$}
        child { node{\lstinline{1}} }
        child { node{\lstinline{2}} }
    }
    child { node {\lstinline{3}} };
\end{tikztree}

\column{0.5\textwidth}
\begin{tikztree}
  \node {$+$}
    child { node {\lstinline{1}} }
    child { node {$+$}
        child { node{\lstinline{2}} }
        child { node{\lstinline{3}} }
    };
\end{tikztree}
\end{columns}
~\\~\\
The result in both cases is 6
\end{frame}
%%%%%%%%%%%%%%%%%%%%%%%%%%%%%%%%%%%%%%%%%%%%%%%%%%%%%%%%%%%%%%%%%%%%%%%%%%%%%%%%

%%%%%%%%%%%%%%%%%%%%%%%%%%%%%%%%%%%%%%%%%%%%%%%%%%%%%%%%%%%%%%%%%%%%%%%%%%%%%%%%
\begin{frame}
\frametitle{Ambiguity}

With ambiguity either parse tree/AST is correct\\~\\

Therefore, the implementer can decide which one they want to use\\~\\

What happens if we use $-$ instead of $+$?
\end{frame}
%%%%%%%%%%%%%%%%%%%%%%%%%%%%%%%%%%%%%%%%%%%%%%%%%%%%%%%%%%%%%%%%%%%%%%%%%%%%%%%%

%%%%%%%%%%%%%%%%%%%%%%%%%%%%%%%%%%%%%%%%%%%%%%%%%%%%%%%%%%%%%%%%%%%%%%%%%%%%%%%%
\begin{frame}
\frametitle{AST Evaluation (2)}

Again, calling \lstinline{eval} with the root node, what is the result?

\begin{columns}[onlytextwidth]
\column{0.5\textwidth}
\begin{tikztree}
  \node {$-$}
    child { node {$-$}
        child { node{\lstinline{1}} }
        child { node{\lstinline{2}} }
    }
    child { node {\lstinline{3}} };
\end{tikztree}

The result is -4
\column{0.5\textwidth}
\begin{tikztree}
  \node {$-$}
    child { node {\lstinline{1}} }
    child { node {$-$}
        child { node{\lstinline{2}} }
        child { node{\lstinline{3}} }
    };
\end{tikztree}

The result is 2
\end{columns}
~\\~\\
Now, depending on the implementation you get different results
\end{frame}
%%%%%%%%%%%%%%%%%%%%%%%%%%%%%%%%%%%%%%%%%%%%%%%%%%%%%%%%%%%%%%%%%%%%%%%%%%%%%%%%

%%%%%%%%%%%%%%%%%%%%%%%%%%%%%%%%%%%%%%%%%%%%%%%%%%%%%%%%%%%%%%%%%%%%%%%%%%%%%%%%
\begin{frame}[plain, noframenumbering]
  \begin{beamercolorbox}[wd=\paperwidth, sep=2em]{solarizedcolorbox}
    {\usebeamerfont{title}\usebeamercolor[fg]{title} End Interpreter \hfill
      Start Unambiguity}
  \end{beamercolorbox}
\end{frame}
%%%%%%%%%%%%%%%%%%%%%%%%%%%%%%%%%%%%%%%%%%%%%%%%%%%%%%%%%%%%%%%%%%%%%%%%%%%%%%%%

%%%%%%%%%%%%%%%%%%%%%%%%%%%%%%%%%%%%%%%%%%%%%%%%%%%%%%%%%%%%%%%%%%%%%%%%%%%%%%%%
\begin{frame}
\frametitle{Removing Ambiguity}

We have to enforce:
\begin{itemize}
  \item Precedence
  \item Associativity
\end{itemize}
\end{frame}
%%%%%%%%%%%%%%%%%%%%%%%%%%%%%%%%%%%%%%%%%%%%%%%%%%%%%%%%%%%%%%%%%%%%%%%%%%%%%%%%

%%%%%%%%%%%%%%%%%%%%%%%%%%%%%%%%%%%%%%%%%%%%%%%%%%%%%%%%%%%%%%%%%%%%%%%%%%%%%%%%
\begin{frame}
\frametitle{Precedence Rules}

\begin{enumerate}
  \item Each level of precedence should be it's own rule\\~\\
  \item The lowest level of precedence should be the top level rule
    (highest level of precedence is the last rule)\\~\\
  \item Each rule should use the next highest level of precedence\\~\\
  \item The highest level of precedence includes the operands
\end{enumerate}
\end{frame}
%%%%%%%%%%%%%%%%%%%%%%%%%%%%%%%%%%%%%%%%%%%%%%%%%%%%%%%%%%%%%%%%%%%%%%%%%%%%%%%%

%%%%%%%%%%%%%%%%%%%%%%%%%%%%%%%%%%%%%%%%%%%%%%%%%%%%%%%%%%%%%%%%%%%%%%%%%%%%%%%%
\begin{frame}
\frametitle{Our Grammar (Attempt 3)}

Following the rules, our grammar is now:
\begin{align*}
T &= \{ +, -, \text{\lstinline{INT}} \}\\
N &= \{ e, p \}\\
S &= e\\
P &= \begin{cases}e \rightarrow e + e \\
     e \rightarrow e - e\\
     e \rightarrow p\\
     p \rightarrow \text{\lstinline{INT}}\end{cases}
\end{align*}
But the grammar is still ambiguous 
\end{frame}
%%%%%%%%%%%%%%%%%%%%%%%%%%%%%%%%%%%%%%%%%%%%%%%%%%%%%%%%%%%%%%%%%%%%%%%%%%%%%%%%

%%%%%%%%%%%%%%%%%%%%%%%%%%%%%%%%%%%%%%%%%%%%%%%%%%%%%%%%%%%%%%%%%%%%%%%%%%%%%%%%
\begin{frame}
\frametitle{Associativity Rules}

Consider $e$ as our current level of precedence and $p$ as the next highest\\~\\

$e \rightarrow e$ \lstinline{OP} $p$, would make it left associative\\~\\

$e \rightarrow p$ \lstinline{OP} $e$, would make it right associative
\end{frame}
%%%%%%%%%%%%%%%%%%%%%%%%%%%%%%%%%%%%%%%%%%%%%%%%%%%%%%%%%%%%%%%%%%%%%%%%%%%%%%%%

%%%%%%%%%%%%%%%%%%%%%%%%%%%%%%%%%%%%%%%%%%%%%%%%%%%%%%%%%%%%%%%%%%%%%%%%%%%%%%%%
\begin{frame}
\frametitle{Our Grammar (Attempt 4)}

We expect left associativity, our grammar is now:
\begin{align*}
T &= \{ +, -, \text{\lstinline{INT}} \}\\
N &= \{ e, p \}\\
S &= e\\
P &= \begin{cases}e \rightarrow e + p \\
     e \rightarrow e - p\\
     e \rightarrow p\\
     p \rightarrow \text{\lstinline{INT}}\end{cases}
\end{align*}
\end{frame}
%%%%%%%%%%%%%%%%%%%%%%%%%%%%%%%%%%%%%%%%%%%%%%%%%%%%%%%%%%%%%%%%%%%%%%%%%%%%%%%%

%%%%%%%%%%%%%%%%%%%%%%%%%%%%%%%%%%%%%%%%%%%%%%%%%%%%%%%%%%%%%%%%%%%%%%%%%%%%%%%%
\begin{frame}
\frametitle{Parsing}

Let's use our previous grammar attempt\\~\\

Parse the following input: \lstinline{"1 - 2 - 3"}\\~\\

We have the following terminals: \lstinline{INT} $-$ \lstinline{INT} $-$
\lstinline{INT}
\end{frame}
%%%%%%%%%%%%%%%%%%%%%%%%%%%%%%%%%%%%%%%%%%%%%%%%%%%%%%%%%%%%%%%%%%%%%%%%%%%%%%%%

%%%%%%%%%%%%%%%%%%%%%%%%%%%%%%%%%%%%%%%%%%%%%%%%%%%%%%%%%%%%%%%%%%%%%%%%%%%%%%%%
\begin{frame}
\frametitle{Previous Derivation}

\begin{align*}
e &\Rightarrow e - p\\
  &\Rightarrow e - p - p\\
  &\Rightarrow p - p - p\\
  &\Rightarrow \text{\lstinline{INT}} - p - p\\
  &\Rightarrow \text{\lstinline{INT}} - \text{\lstinline{INT}} - p\\
  &\Rightarrow \text{\lstinline{INT}} - \text{\lstinline{INT}} -
               \text{\lstinline{INT}}
\end{align*}

This is the only derivation
\end{frame}
%%%%%%%%%%%%%%%%%%%%%%%%%%%%%%%%%%%%%%%%%%%%%%%%%%%%%%%%%%%%%%%%%%%%%%%%%%%%%%%%

%%%%%%%%%%%%%%%%%%%%%%%%%%%%%%%%%%%%%%%%%%%%%%%%%%%%%%%%%%%%%%%%%%%%%%%%%%%%%%%%
\begin{frame}
\frametitle{Previous Parse Tree}

\begin{tikztree}
  \node {$e$}
    child { node {$e$} 
      child { node {$e$}
        child { node{$p$}
          child { node{\lstinline{INT}} }
        }
      }
      child { node {$-$} }
      child { node {$p$} child { node{\lstinline{INT}} }}
    }
    child { node {$-$} }
    child { node {$p$}
      child { node{\lstinline{INT}} }
    };
\end{tikztree}

What about the corresponding AST?
\end{frame}
%%%%%%%%%%%%%%%%%%%%%%%%%%%%%%%%%%%%%%%%%%%%%%%%%%%%%%%%%%%%%%%%%%%%%%%%%%%%%%%%

%%%%%%%%%%%%%%%%%%%%%%%%%%%%%%%%%%%%%%%%%%%%%%%%%%%%%%%%%%%%%%%%%%%%%%%%%%%%%%%%
\begin{frame}
\frametitle{Previous AST}

\begin{tikztree}
  \node {$-$}
    child { node {$-$}
        child { node{\lstinline{1}} }
        child { node{\lstinline{2}} }
    }
    child { node {\lstinline{3}} };
\end{tikztree}

Our interpreter would return \lstinline{-4} as expected
\end{frame}
%%%%%%%%%%%%%%%%%%%%%%%%%%%%%%%%%%%%%%%%%%%%%%%%%%%%%%%%%%%%%%%%%%%%%%%%%%%%%%%%

%%%%%%%%%%%%%%%%%%%%%%%%%%%%%%%%%%%%%%%%%%%%%%%%%%%%%%%%%%%%%%%%%%%%%%%%%%%%%%%%
\begin{frame}
\frametitle{Our Grammar (Attempt 5)}

Let's add multiplication and division:

\begin{align*}
T &= \{ +, -, *, /, \text{\lstinline{INT}} \}\\
N &= \{ e, p \}\\
S &= e\\
P &= \begin{cases}e \rightarrow e + p \\
     e \rightarrow e - p\\
     e \rightarrow e * p\\
     e \rightarrow e / p\\
     e \rightarrow p\\
     p \rightarrow \text{\lstinline{INT}}\end{cases}
\end{align*}
\end{frame}
%%%%%%%%%%%%%%%%%%%%%%%%%%%%%%%%%%%%%%%%%%%%%%%%%%%%%%%%%%%%%%%%%%%%%%%%%%%%%%%%

%%%%%%%%%%%%%%%%%%%%%%%%%%%%%%%%%%%%%%%%%%%%%%%%%%%%%%%%%%%%%%%%%%%%%%%%%%%%%%%%
\begin{frame}
\frametitle{Parsing}

Let's use our previous grammar attempt\\~\\

Parse the following input: \lstinline{"1 - 2 * 3"}\\~\\

We have the following terminals: \lstinline{INT} $-$ \lstinline{INT} $*$
\lstinline{INT}
\end{frame}
%%%%%%%%%%%%%%%%%%%%%%%%%%%%%%%%%%%%%%%%%%%%%%%%%%%%%%%%%%%%%%%%%%%%%%%%%%%%%%%%

%%%%%%%%%%%%%%%%%%%%%%%%%%%%%%%%%%%%%%%%%%%%%%%%%%%%%%%%%%%%%%%%%%%%%%%%%%%%%%%%
\begin{frame}
\frametitle{Previous Derivation}

\begin{align*}
e &\Rightarrow e * p\\
  &\Rightarrow e - p * p\\
  &\Rightarrow p - p * p\\
  &\Rightarrow \text{\lstinline{INT}} - p * p\\
  &\Rightarrow \text{\lstinline{INT}} - \text{\lstinline{INT}} * p\\
  &\Rightarrow \text{\lstinline{INT}} - \text{\lstinline{INT}} *
               \text{\lstinline{INT}}
\end{align*}
\end{frame}
%%%%%%%%%%%%%%%%%%%%%%%%%%%%%%%%%%%%%%%%%%%%%%%%%%%%%%%%%%%%%%%%%%%%%%%%%%%%%%%%

%%%%%%%%%%%%%%%%%%%%%%%%%%%%%%%%%%%%%%%%%%%%%%%%%%%%%%%%%%%%%%%%%%%%%%%%%%%%%%%%
\begin{frame}
\frametitle{Previous Parse Tree}

\begin{tikztree}
  \node {$e$}
    child { node {$e$} 
      child { node {$e$}
        child { node{$p$}
          child { node{\lstinline{INT}} }
        }
      }
      child { node {$-$} }
      child { node {$p$} child { node{\lstinline{INT}} }}
    }
    child { node {$*$} }
    child { node {$p$}
      child { node{\lstinline{INT}} }
    };
\end{tikztree}
\end{frame}
%%%%%%%%%%%%%%%%%%%%%%%%%%%%%%%%%%%%%%%%%%%%%%%%%%%%%%%%%%%%%%%%%%%%%%%%%%%%%%%%

%%%%%%%%%%%%%%%%%%%%%%%%%%%%%%%%%%%%%%%%%%%%%%%%%%%%%%%%%%%%%%%%%%%%%%%%%%%%%%%%
\begin{frame}
\frametitle{Previous AST}

\begin{tikztree}
  \node {$*$}
    child { node {$-$}
        child { node{\lstinline{1}} }
        child { node{\lstinline{2}} }
    }
    child { node {\lstinline{3}} };
\end{tikztree}

Our interpreter would return \lstinline{-3} which is wrong\\~\\

How would we fix this?
\end{frame}
%%%%%%%%%%%%%%%%%%%%%%%%%%%%%%%%%%%%%%%%%%%%%%%%%%%%%%%%%%%%%%%%%%%%%%%%%%%%%%%%

%%%%%%%%%%%%%%%%%%%%%%%%%%%%%%%%%%%%%%%%%%%%%%%%%%%%%%%%%%%%%%%%%%%%%%%%%%%%%%%%
\begin{frame}
\frametitle{Precedence Rules}

\begin{enumerate}
  \item Each level of precedence should be it's own rule\\~\\
  \item The lowest level of precedence should be the top level rule
    (highest level of precedence is the last rule)\\~\\
  \item Each rule should use the next highest level of precedence\\~\\
  \item The highest level of precedence includes the operands
\end{enumerate}
\end{frame}
%%%%%%%%%%%%%%%%%%%%%%%%%%%%%%%%%%%%%%%%%%%%%%%%%%%%%%%%%%%%%%%%%%%%%%%%%%%%%%%%

%%%%%%%%%%%%%%%%%%%%%%%%%%%%%%%%%%%%%%%%%%%%%%%%%%%%%%%%%%%%%%%%%%%%%%%%%%%%%%%%
\begin{frame}
\frametitle{Our Grammar (Attempt 6)}

We have to add another level of precedence ($t$) for multiplication and division:
\begin{align*}
T &= \{ +, -, *, /, \text{\lstinline{INT}} \}\\
N &= \{ e, t, p \}\\
S &= e\\
P &= \begin{cases}e \rightarrow e + t\\
     e \rightarrow e - t\\
     e \rightarrow t\\
     t \rightarrow t * p\\
     t \rightarrow t / p\\
     t \rightarrow p\\
     p \rightarrow \text{\lstinline{INT}}\end{cases}
\end{align*}
\end{frame}
%%%%%%%%%%%%%%%%%%%%%%%%%%%%%%%%%%%%%%%%%%%%%%%%%%%%%%%%%%%%%%%%%%%%%%%%%%%%%%%%

%%%%%%%%%%%%%%%%%%%%%%%%%%%%%%%%%%%%%%%%%%%%%%%%%%%%%%%%%%%%%%%%%%%%%%%%%%%%%%%%
\begin{frame}
\frametitle{Parsing}

Let's use our previous grammar attempt\\~\\

Parse the following input: \lstinline{"1 - 2 * 3"}\\~\\

We have the following terminals: \lstinline{INT} $-$ \lstinline{INT} $*$
\lstinline{INT}
\end{frame}
%%%%%%%%%%%%%%%%%%%%%%%%%%%%%%%%%%%%%%%%%%%%%%%%%%%%%%%%%%%%%%%%%%%%%%%%%%%%%%%%

%%%%%%%%%%%%%%%%%%%%%%%%%%%%%%%%%%%%%%%%%%%%%%%%%%%%%%%%%%%%%%%%%%%%%%%%%%%%%%%%
\begin{frame}
\frametitle{Previous Derivation}

\begin{align*}
e &\Rightarrow e - t\\
  &\Rightarrow t - t\\
  &\Rightarrow p - t\\
  &\Rightarrow \text{\lstinline{INT}} - t\\
  &\Rightarrow \text{\lstinline{INT}} - t * p\\
  &\Rightarrow \text{\lstinline{INT}} - p * p\\
  &\Rightarrow \text{\lstinline{INT}} - \text{\lstinline{INT}} * p\\
  &\Rightarrow \text{\lstinline{INT}} - \text{\lstinline{INT}} *
               \text{\lstinline{INT}}
\end{align*}
\end{frame}
%%%%%%%%%%%%%%%%%%%%%%%%%%%%%%%%%%%%%%%%%%%%%%%%%%%%%%%%%%%%%%%%%%%%%%%%%%%%%%%%

%%%%%%%%%%%%%%%%%%%%%%%%%%%%%%%%%%%%%%%%%%%%%%%%%%%%%%%%%%%%%%%%%%%%%%%%%%%%%%%%
\begin{frame}
\frametitle{Previous Parse Tree}

\scalebox{0.8}{\begin{minipage}{1.25\textwidth}
\begin{tikztree}
  \node {$e$}
    child { node {$e$} 
      child { node {$t$}
        child { node {$p$}
          child { node {\lstinline{INT}} }
        }
      }
    }
    child { node {$-$} }
    child { node {$t$}
      child { node{$t$}
        child { node {$p$}
          child { node {\lstinline{INT}} }
        }
      }
      child { node{$*$} }
      child { node{$p$}
          child { node {\lstinline{INT}} }
      }
    };
\end{tikztree}
\end{minipage}}
\end{frame}
%%%%%%%%%%%%%%%%%%%%%%%%%%%%%%%%%%%%%%%%%%%%%%%%%%%%%%%%%%%%%%%%%%%%%%%%%%%%%%%%

%%%%%%%%%%%%%%%%%%%%%%%%%%%%%%%%%%%%%%%%%%%%%%%%%%%%%%%%%%%%%%%%%%%%%%%%%%%%%%%%
\begin{frame}
\frametitle{Previous AST}

\begin{tikztree}
  \node {$-$}
    child { node {\lstinline{1}} }
    child { node {$*$}
        child { node{\lstinline{2}} }
        child { node{\lstinline{3}} }
    };
\end{tikztree}

Our interpreter would return \lstinline{-5} which is correct
\end{frame}
%%%%%%%%%%%%%%%%%%%%%%%%%%%%%%%%%%%%%%%%%%%%%%%%%%%%%%%%%%%%%%%%%%%%%%%%%%%%%%%%

%%%%%%%%%%%%%%%%%%%%%%%%%%%%%%%%%%%%%%%%%%%%%%%%%%%%%%%%%%%%%%%%%%%%%%%%%%%%%%%%
\begin{frame}
\frametitle{Grouping and Unary Precedence}

Let's add grouping $()$ and a unary $-$\\~\\

Grouping is always done at the highest level of precedence and calls the lowest
level precedence rule\\~\\

Unary operators follow the same rules as others, except there is no
associativity (so it should call itself)
\end{frame}
%%%%%%%%%%%%%%%%%%%%%%%%%%%%%%%%%%%%%%%%%%%%%%%%%%%%%%%%%%%%%%%%%%%%%%%%%%%%%%%%

%%%%%%%%%%%%%%%%%%%%%%%%%%%%%%%%%%%%%%%%%%%%%%%%%%%%%%%%%%%%%%%%%%%%%%%%%%%%%%%%
\begin{frame}
\frametitle{Our Completed Grammar}

\begin{align*}
T &= \{ +, -, *, /, (, ), \text{\lstinline{INT}} \}\\
N &= \{ e, t, f, p \}\\
S &= e\\
P &= \begin{cases}e \rightarrow e + t\\
     e \rightarrow e - t\\
     e \rightarrow t\\
     t \rightarrow t * f\\
     t \rightarrow t / f\\
     t \rightarrow f\\
     f \rightarrow - f\\
     f \rightarrow p\\
     p \rightarrow ( e )\\
     p \rightarrow \text{\lstinline{INT}}\end{cases}
\end{align*}
\end{frame}
%%%%%%%%%%%%%%%%%%%%%%%%%%%%%%%%%%%%%%%%%%%%%%%%%%%%%%%%%%%%%%%%%%%%%%%%%%%%%%%%

%%%%%%%%%%%%%%%%%%%%%%%%%%%%%%%%%%%%%%%%%%%%%%%%%%%%%%%%%%%%%%%%%%%%%%%%%%%%%%%%
\begin{frame}
\frametitle{Grammar Question}

Consider the grammar:
\begin{align*}
T &= \{ \text{\lstinline{IF}}, \text{\lstinline{ELSE}} \}\\
N &= \{ ifstmt, expr, stmt \}\\
S &= stmt\\
P &= \begin{cases}stmt \rightarrow ifstmt\\
     ifstmt \rightarrow \text{\lstinline{IF}} \; expr \; stmt\\
     ifstmt \rightarrow \text{\lstinline{IF}} \; expr \; stmt \;
                        \text{\lstinline{ELSE}} \; stmt\end{cases}
\end{align*}

Is this ambiguous?
\end{frame}
%%%%%%%%%%%%%%%%%%%%%%%%%%%%%%%%%%%%%%%%%%%%%%%%%%%%%%%%%%%%%%%%%%%%%%%%%%%%%%%%

%%%%%%%%%%%%%%%%%%%%%%%%%%%%%%%%%%%%%%%%%%%%%%%%%%%%%%%%%%%%%%%%%%%%%%%%%%%%%%%%
\begin{frame}
\frametitle{Grammar Answer}

Yes, it is ambiguous\\~\\

Consider the \alert{sentential form}:\\
\lstinline{IF} $expr$ \lstinline{IF} $expr$ $stmt$ \lstinline{ELSE}
$stmt$\\~\\

We have two leftmost derivations of this form
\end{frame}
%%%%%%%%%%%%%%%%%%%%%%%%%%%%%%%%%%%%%%%%%%%%%%%%%%%%%%%%%%%%%%%%%%%%%%%%%%%%%%%%

%%%%%%%%%%%%%%%%%%%%%%%%%%%%%%%%%%%%%%%%%%%%%%%%%%%%%%%%%%%%%%%%%%%%%%%%%%%%%%%%
\begin{frame}
\frametitle{Previous Derivations}

\begin{align*}
stmt &\Rightarrow ifstmt\\
     &\Rightarrow \text{\lstinline{IF}} \; expr \; stmt\\
     &\Rightarrow \text{\lstinline{IF}} \; expr \; ifstmt\\
     &\Rightarrow \text{\lstinline{IF}} \; expr \; \text{\lstinline{IF}} \; expr
                  \; stmt \; \text{\lstinline{ELSE}} \; stmt\\
\end{align*}
\begin{align*}
stmt &\Rightarrow ifstmt\\
     &\Rightarrow \text{\lstinline{IF}} \; expr \; stmt \;
                  \text{\lstinline{ELSE}} \; stmt\\
     &\Rightarrow \text{\lstinline{IF}} \; expr \; ifstmt \;
                  \text{\lstinline{ELSE}} \; stmt\\
     &\Rightarrow \text{\lstinline{IF}} \; expr \; \text{\lstinline{IF}} \; expr
                  \; stmt \; \text{\lstinline{ELSE}} \; stmt\\
\end{align*}
This is called the dangling else problem
\end{frame}
%%%%%%%%%%%%%%%%%%%%%%%%%%%%%%%%%%%%%%%%%%%%%%%%%%%%%%%%%%%%%%%%%%%%%%%%%%%%%%%%

%%%%%%%%%%%%%%%%%%%%%%%%%%%%%%%%%%%%%%%%%%%%%%%%%%%%%%%%%%%%%%%%%%%%%%%%%%%%%%%%
\begin{frame}[fragile]
\frametitle{Dangling Else Example}

Consider the following Java code:

\begin{lstlisting}[backgroundcolor=\color{solarizedRebase02}]
if (x > 0) if (x > 3) x = 3 else x = 0
\end{lstlisting}
~\\
When \lstinline{x = 1} the final value of \lstinline{x} depends on our parse
\end{frame}
%%%%%%%%%%%%%%%%%%%%%%%%%%%%%%%%%%%%%%%%%%%%%%%%%%%%%%%%%%%%%%%%%%%%%%%%%%%%%%%%

%%%%%%%%%%%%%%%%%%%%%%%%%%%%%%%%%%%%%%%%%%%%%%%%%%%%%%%%%%%%%%%%%%%%%%%%%%%%%%%%
\begin{frame}
\frametitle{Removing the Dangling Else}

There are 4 main approaches:
\vspace{1em}
\begin{itemize}
  \item Add an \lstinline{endif} statement (like in VHDL)\\~\\
  \item Specify the meaning in the language specification and leave it up the
    implementations (like Java, which always does inner)\\~\\
  \item Introduce indenting into the grammar (like Python)\\~\\
  \item Require every \lstinline{if} to have an \lstinline{else} (like Haskell)
\end{itemize}
\end{frame}
%%%%%%%%%%%%%%%%%%%%%%%%%%%%%%%%%%%%%%%%%%%%%%%%%%%%%%%%%%%%%%%%%%%%%%%%%%%%%%%%

%%%%%%%%%%%%%%%%%%%%%%%%%%%%%%%%%%%%%%%%%%%%%%%%%%%%%%%%%%%%%%%%%%%%%%%%%%%%%%%%
\begin{frame}[plain, noframenumbering]
  \begin{beamercolorbox}[wd=\paperwidth, sep=2em]{solarizedcolorbox}
    {\usebeamerfont{title}\usebeamercolor[fg]{title} End Unambiguity \hfill
      Start Techniques}
  \end{beamercolorbox}
\end{frame}
%%%%%%%%%%%%%%%%%%%%%%%%%%%%%%%%%%%%%%%%%%%%%%%%%%%%%%%%%%%%%%%%%%%%%%%%%%%%%%%%

%%%%%%%%%%%%%%%%%%%%%%%%%%%%%%%%%%%%%%%%%%%%%%%%%%%%%%%%%%%%%%%%%%%%%%%%%%%%%%%%
\begin{frame}
\frametitle{Top-Down Parsing}

Consider the grammar:
\begin{align*}
T &= \{ \text{\lstinline{0}}, \text{\lstinline{1}} \}\\
N &= \{ s, b \}\\
S &= s\\
P &= \begin{cases}s \rightarrow \text{\lstinline{1}} b \text{\lstinline{0}}\\
  s \rightarrow \text{\lstinline{0}} b \text{\lstinline{1}}\\
  b \rightarrow \text{\lstinline{1}} \text{\lstinline{0}}\\
  b \rightarrow \text{\lstinline{1}} \text{\lstinline{1}}\end{cases}
\end{align*}

Let's do our leftmost derivation one token at a time\\~\\

This is called \alert{top-down parsing}
\end{frame}
%%%%%%%%%%%%%%%%%%%%%%%%%%%%%%%%%%%%%%%%%%%%%%%%%%%%%%%%%%%%%%%%%%%%%%%%%%%%%%%%

%%%%%%%%%%%%%%%%%%%%%%%%%%%%%%%%%%%%%%%%%%%%%%%%%%%%%%%%%%%%%%%%%%%%%%%%%%%%%%%%
\begin{frame}
\frametitle{Top-Down Parse (1)}

Current token: \lstinline{1}
\begin{align*}
s &\Rightarrow \text{\lstinline{1}} b \text{\lstinline{0}}
\end{align*}
\end{frame}
%%%%%%%%%%%%%%%%%%%%%%%%%%%%%%%%%%%%%%%%%%%%%%%%%%%%%%%%%%%%%%%%%%%%%%%%%%%%%%%%

%%%%%%%%%%%%%%%%%%%%%%%%%%%%%%%%%%%%%%%%%%%%%%%%%%%%%%%%%%%%%%%%%%%%%%%%%%%%%%%%
\begin{frame}
\frametitle{Top-Down Parse (2)}

Current token: \lstinline{1}
\begin{align*}
s &\Rightarrow \text{\lstinline{1}} b \text{\lstinline{0}}\\
s &\Rightarrow \text{\lstinline{1}} \text{\lstinline{1}} \text{\lstinline{0}}
               \text{\lstinline{0}}
\end{align*}

This is a guess since we don't know the next token
\end{frame}
%%%%%%%%%%%%%%%%%%%%%%%%%%%%%%%%%%%%%%%%%%%%%%%%%%%%%%%%%%%%%%%%%%%%%%%%%%%%%%%%

%%%%%%%%%%%%%%%%%%%%%%%%%%%%%%%%%%%%%%%%%%%%%%%%%%%%%%%%%%%%%%%%%%%%%%%%%%%%%%%%
\begin{frame}
\frametitle{Top-Down Parse (3)}

Current token: \lstinline{1}\\~\\

Our previous guess was wrong, so we backtrack to
\begin{align*}
s &\Rightarrow \text{\lstinline{1}} b \text{\lstinline{0}}\\
\end{align*}
And guess again
\begin{align*}
s &\Rightarrow \text{\lstinline{1}} b \text{\lstinline{0}}\\
s &\Rightarrow \text{\lstinline{1}} \text{\lstinline{1}} \text{\lstinline{1}}
               \text{\lstinline{0}}
\end{align*}
\end{frame}
%%%%%%%%%%%%%%%%%%%%%%%%%%%%%%%%%%%%%%%%%%%%%%%%%%%%%%%%%%%%%%%%%%%%%%%%%%%%%%%%

%%%%%%%%%%%%%%%%%%%%%%%%%%%%%%%%%%%%%%%%%%%%%%%%%%%%%%%%%%%%%%%%%%%%%%%%%%%%%%%%
\begin{frame}
\frametitle{Top-Down Parse (4)}

Current token: \lstinline{0}\\~\\

\begin{align*}
s &\Rightarrow \text{\lstinline{1}} b \text{\lstinline{0}}\\
s &\Rightarrow \text{\lstinline{1}} \text{\lstinline{1}} \text{\lstinline{1}}
               \text{\lstinline{0}}
\end{align*}

This is the last token\\~\\

Therefore \lstinline{1110} is valid in our grammar
\end{frame}
%%%%%%%%%%%%%%%%%%%%%%%%%%%%%%%%%%%%%%%%%%%%%%%%%%%%%%%%%%%%%%%%%%%%%%%%%%%%%%%%

%%%%%%%%%%%%%%%%%%%%%%%%%%%%%%%%%%%%%%%%%%%%%%%%%%%%%%%%%%%%%%%%%%%%%%%%%%%%%%%%
\begin{frame}
\frametitle{Recursive Descent Parsing}

An example of a top-down parser is a \alert{recursive descent parser}\\~\\

Translating the grammar follows these rules:
\begin{itemize}
  \item Each non-terminal turns into a function, the body turns into the
    production
  \item Each non-terminal in the production is a function call to the
    corresponding non-terminal
  \item Each terminal in the production is a call to the lexer's \lstinline{consume} with
    the terminal as the argument\\~\\
\end{itemize}

You can use \lstinline{inspect} with a terminal as an argument to check the
current token and decide which rule to follow\\~\\

There could be backtracking as well if you need it
\end{frame}
%%%%%%%%%%%%%%%%%%%%%%%%%%%%%%%%%%%%%%%%%%%%%%%%%%%%%%%%%%%%%%%%%%%%%%%%%%%%%%%%

%%%%%%%%%%%%%%%%%%%%%%%%%%%%%%%%%%%%%%%%%%%%%%%%%%%%%%%%%%%%%%%%%%%%%%%%%%%%%%%%
\begin{frame}[fragile]
\frametitle{Implementing Our Grammar}

Let's implement our first rule $e \rightarrow e + t$\\~\\

One implementation is the following:

\begin{lstlisting}[backgroundcolor=\color{solarizedRebase02}]
public void E() {
  E();
  consume("+");
  T();
}
\end{lstlisting}
~\\
Do you see the major problem with this?
\end{frame}
%%%%%%%%%%%%%%%%%%%%%%%%%%%%%%%%%%%%%%%%%%%%%%%%%%%%%%%%%%%%%%%%%%%%%%%%%%%%%%%%

%%%%%%%%%%%%%%%%%%%%%%%%%%%%%%%%%%%%%%%%%%%%%%%%%%%%%%%%%%%%%%%%%%%%%%%%%%%%%%%%
\begin{frame}
\frametitle{Removing Left Recursion}

Left recursion leads to infinite recursion in our function\\~\\

We can remove left recursion in grammars of the form:
\begin{align*}
P &= \begin{cases}a \rightarrow a x\\
  a \rightarrow y\end{cases}
\end{align*}

where $a$ is a non-terminal, $x$ and $y$ are any sequence of
terminals/non-terminals\\~\\

By replacing it with:
\begin{align*}
P &= \begin{cases}a \rightarrow y a'\\
  a' \rightarrow x a'\\
  a' \rightarrow \varepsilon \text{} \end{cases}
\end{align*}
Note that $\varepsilon$ represents no symbols
\end{frame}
%%%%%%%%%%%%%%%%%%%%%%%%%%%%%%%%%%%%%%%%%%%%%%%%%%%%%%%%%%%%%%%%%%%%%%%%%%%%%%%%

%%%%%%%%%%%%%%%%%%%%%%%%%%%%%%%%%%%%%%%%%%%%%%%%%%%%%%%%%%%%%%%%%%%%%%%%%%%%%%%%
\begin{frame}
\frametitle{Removing Left Recursion Question}

Remove left recursion from our rule:
\begin{align*}
P &= \begin{cases}e \rightarrow e + t\\
  e \rightarrow t\end{cases}
\end{align*}
\end{frame}
%%%%%%%%%%%%%%%%%%%%%%%%%%%%%%%%%%%%%%%%%%%%%%%%%%%%%%%%%%%%%%%%%%%%%%%%%%%%%%%%

%%%%%%%%%%%%%%%%%%%%%%%%%%%%%%%%%%%%%%%%%%%%%%%%%%%%%%%%%%%%%%%%%%%%%%%%%%%%%%%%
\begin{frame}
\frametitle{Removing Left Recursion Answer}

Following the rules we would replace it with:
\begin{align*}
P &= \begin{cases}e \rightarrow t \; e'\\
  e' \rightarrow + \; t \; e'\\
  e' \rightarrow \varepsilon \text{} \end{cases}
\end{align*}
\end{frame}
%%%%%%%%%%%%%%%%%%%%%%%%%%%%%%%%%%%%%%%%%%%%%%%%%%%%%%%%%%%%%%%%%%%%%%%%%%%%%%%%

%%%%%%%%%%%%%%%%%%%%%%%%%%%%%%%%%%%%%%%%%%%%%%%%%%%%%%%%%%%%%%%%%%%%%%%%%%%%%%%%
\begin{frame}[plain, noframenumbering]
  \begin{beamercolorbox}[wd=\paperwidth, sep=2em]{solarizedcolorbox}
    {\usebeamerfont{title}\usebeamercolor[fg]{title} End Techniques \hfill Start
      EBNF}
  \end{beamercolorbox}
\end{frame}
%%%%%%%%%%%%%%%%%%%%%%%%%%%%%%%%%%%%%%%%%%%%%%%%%%%%%%%%%%%%%%%%%%%%%%%%%%%%%%%%

%%%%%%%%%%%%%%%%%%%%%%%%%%%%%%%%%%%%%%%%%%%%%%%%%%%%%%%%%%%%%%%%%%%%%%%%%%%%%%%%
\begin{frame}
\frametitle{Grammar Equivalence}

Let's demonstrate that the grammars are equivalent:
\begin{align*}
P &= \begin{cases}a \rightarrow a x\\
  a \rightarrow y\end{cases}
\end{align*}

\begin{align*}
P &= \begin{cases}a \rightarrow y a'\\
  a' \rightarrow x a'\\
  a' \rightarrow \varepsilon \text{} \end{cases}
\end{align*}

What are some example derivations?
\end{frame}
%%%%%%%%%%%%%%%%%%%%%%%%%%%%%%%%%%%%%%%%%%%%%%%%%%%%%%%%%%%%%%%%%%%%%%%%%%%%%%%%

%%%%%%%%%%%%%%%%%%%%%%%%%%%%%%%%%%%%%%%%%%%%%%%%%%%%%%%%%%%%%%%%%%%%%%%%%%%%%%%%
\begin{frame}
\frametitle{Extended Backus-Naur Form}

$y$, $yx$, $yxx$, $yxxx$\\~\\

If $x$ and $y$ were just characters, this would be the same as the regular
expression \lstinline{yx*}\\~\\

\alert{Extended Backus-Naur Form} (EBNF) is just that, it's BNF with regular
expression operations (\lstinline{+}, \lstinline{*}, \lstinline{()},
\lstinline{?})
\end{frame}
%%%%%%%%%%%%%%%%%%%%%%%%%%%%%%%%%%%%%%%%%%%%%%%%%%%%%%%%%%%%%%%%%%%%%%%%%%%%%%%%

%%%%%%%%%%%%%%%%%%%%%%%%%%%%%%%%%%%%%%%%%%%%%%%%%%%%%%%%%%%%%%%%%%%%%%%%%%%%%%%%
\begin{frame}
\frametitle{EBNF Example}

Assume the grammar:
\begin{align*}
T &= \{ \text{\lstinline{A}}, \text{\lstinline{B}}, \text{\lstinline{C}} \}\\
N &= \{ a, x \}\\
S &= a\\
P &= \begin{cases}a \rightarrow \text{\lstinline{A}} \; x \;
     \text{\lstinline{C}}\end{cases}
\end{align*}

Let's consider some variations for the rule $x$ in BNF and EBNF
\end{frame}
%%%%%%%%%%%%%%%%%%%%%%%%%%%%%%%%%%%%%%%%%%%%%%%%%%%%%%%%%%%%%%%%%%%%%%%%%%%%%%%%

%%%%%%%%%%%%%%%%%%%%%%%%%%%%%%%%%%%%%%%%%%%%%%%%%%%%%%%%%%%%%%%%%%%%%%%%%%%%%%%%
\begin{frame}
\frametitle{EBNF Variation (1)}

The following productions are equivalent:
\begin{align*}
P &= \begin{cases}a \rightarrow \text{\lstinline{A}} \; x \;
     \text{\lstinline{C}}\\
     x \rightarrow x \; \text{\lstinline{B}}\\
     x \rightarrow \varepsilon \end{cases}
\end{align*}
\begin{align*}
P &= \begin{cases}a \rightarrow \text{\lstinline{A}} \; x \;
     \text{\lstinline{C}}\\
     x \rightarrow \text{\lstinline{B}}* \end{cases}
\end{align*}
\begin{align*}
P &= \begin{cases}a \rightarrow \text{\lstinline{A}} \; \text{\lstinline{B}}* \;
     \text{\lstinline{C}}\end{cases}
\end{align*}
\end{frame}
%%%%%%%%%%%%%%%%%%%%%%%%%%%%%%%%%%%%%%%%%%%%%%%%%%%%%%%%%%%%%%%%%%%%%%%%%%%%%%%%

%%%%%%%%%%%%%%%%%%%%%%%%%%%%%%%%%%%%%%%%%%%%%%%%%%%%%%%%%%%%%%%%%%%%%%%%%%%%%%%%
\begin{frame}
\frametitle{EBNF Variation (2)}

The following productions are equivalent:
\begin{align*}
P &= \begin{cases}a \rightarrow \text{\lstinline{A}} \; x \;
     \text{\lstinline{C}}\\
     x \rightarrow \text{\lstinline{B}}\\
     x \rightarrow \varepsilon \end{cases}
\end{align*}
\begin{align*}
P &= \begin{cases}a \rightarrow \text{\lstinline{A}} \; x \;
     \text{\lstinline{C}}\\
     x \rightarrow \text{\lstinline{B}}? \end{cases}
\end{align*}
\end{frame}
%%%%%%%%%%%%%%%%%%%%%%%%%%%%%%%%%%%%%%%%%%%%%%%%%%%%%%%%%%%%%%%%%%%%%%%%%%%%%%%%

%%%%%%%%%%%%%%%%%%%%%%%%%%%%%%%%%%%%%%%%%%%%%%%%%%%%%%%%%%%%%%%%%%%%%%%%%%%%%%%%
\begin{frame}
\frametitle{EBNF Variation (3)}

The following productions are equivalent:
\begin{align*}
P &= \begin{cases}a \rightarrow \text{\lstinline{A}} \; x \;
     \text{\lstinline{C}}\\
     x \rightarrow \text{\lstinline{B}} y\\
     y \rightarrow \text{\lstinline{B}} y\\
     y \rightarrow \varepsilon \end{cases}
\end{align*}
\begin{align*}
P &= \begin{cases}a \rightarrow \text{\lstinline{A}} \; x \;
     \text{\lstinline{C}}\\
     x \rightarrow \text{\lstinline{B}}+ \end{cases}
\end{align*}
\end{frame}
%%%%%%%%%%%%%%%%%%%%%%%%%%%%%%%%%%%%%%%%%%%%%%%%%%%%%%%%%%%%%%%%%%%%%%%%%%%%%%%%

%%%%%%%%%%%%%%%%%%%%%%%%%%%%%%%%%%%%%%%%%%%%%%%%%%%%%%%%%%%%%%%%%%%%%%%%%%%%%%%%
\begin{frame}[fragile]
\frametitle{EBNF Implementation Example}

For recursive descent parsers $*$ translates to a \lstinline{while} loop, $?$
translates to an \lstinline{if} statement
\begin{align*}
P &= \begin{cases}x \rightarrow \text{\lstinline{B}}* \end{cases}
\end{align*}
\begin{lstlisting}[backgroundcolor=\color{solarizedRebase02}]
public void x() {
  while (inspect("B")) { consume("B"); }
}
\end{lstlisting}
\begin{align*}
P &= \begin{cases}x \rightarrow \text{\lstinline{B}}? \end{cases}
\end{align*}
\begin{lstlisting}[backgroundcolor=\color{solarizedRebase02}]
public void x() {
  if (inspect("B")) { consume("B"); }
}
\end{lstlisting}
\end{frame}
%%%%%%%%%%%%%%%%%%%%%%%%%%%%%%%%%%%%%%%%%%%%%%%%%%%%%%%%%%%%%%%%%%%%%%%%%%%%%%%%

%%%%%%%%%%%%%%%%%%%%%%%%%%%%%%%%%%%%%%%%%%%%%%%%%%%%%%%%%%%%%%%%%%%%%%%%%%%%%%%%
\begin{frame}
\frametitle{EBNF Question}

Consider our rewritten productions:
\begin{align*}
P &= \begin{cases}e \rightarrow t \; e'\\
  e' \rightarrow + \; t \; e'\\
  e' \rightarrow \varepsilon \text{} \end{cases}
\end{align*}
Let's convert this grammar to EBNF
\end{frame}
%%%%%%%%%%%%%%%%%%%%%%%%%%%%%%%%%%%%%%%%%%%%%%%%%%%%%%%%%%%%%%%%%%%%%%%%%%%%%%%%

%%%%%%%%%%%%%%%%%%%%%%%%%%%%%%%%%%%%%%%%%%%%%%%%%%%%%%%%%%%%%%%%%%%%%%%%%%%%%%%%
\begin{frame}
\frametitle{EBNF Answer}

Recognize that $e'$ can be transformed:
\begin{align*}
P &= \begin{cases}e \rightarrow t \; (+ \; t)*\end{cases}
\end{align*}
\end{frame}
%%%%%%%%%%%%%%%%%%%%%%%%%%%%%%%%%%%%%%%%%%%%%%%%%%%%%%%%%%%%%%%%%%%%%%%%%%%%%%%%

%%%%%%%%%%%%%%%%%%%%%%%%%%%%%%%%%%%%%%%%%%%%%%%%%%%%%%%%%%%%%%%%%%%%%%%%%%%%%%%%
\begin{frame}
\frametitle{Converting Our Grammar to EBNF}

We also have a $-$ at that level of precedence:
\begin{align*}
P &= \begin{cases}e \rightarrow t \; (+ \; t)*\\
     e \rightarrow t \; (- \; t)*\end{cases}
\end{align*}

We can factor out the common parts of both productions
\begin{align*}
P &= \begin{cases}e \rightarrow t \; ((+ \; t) \; | \; (- \; t))*\end{cases}
\end{align*}
\begin{align*}
P &= \begin{cases}e \rightarrow t \; ((+|-) \; t)*\end{cases}
\end{align*}

The production is now in a form that's easily translated into a recursive
descent parser
\end{frame}
%%%%%%%%%%%%%%%%%%%%%%%%%%%%%%%%%%%%%%%%%%%%%%%%%%%%%%%%%%%%%%%%%%%%%%%%%%%%%%%%

%%%%%%%%%%%%%%%%%%%%%%%%%%%%%%%%%%%%%%%%%%%%%%%%%%%%%%%%%%%%%%%%%%%%%%%%%%%%%%%%
\begin{frame}
\frametitle{EBNF for Associativity}

Let's specify our previous associativity rules in EBNF\\~\\

Consider $e$ as the current level of precedence and $p$ is the next highest
level of precedence\\~\\

$e \rightarrow p \;($\lstinline{OP} $p )*$ is left associative\\~\\

$e \rightarrow p \;($\lstinline{OP} $e )?$ is right associative
\end{frame}
%%%%%%%%%%%%%%%%%%%%%%%%%%%%%%%%%%%%%%%%%%%%%%%%%%%%%%%%%%%%%%%%%%%%%%%%%%%%%%%%

%%%%%%%%%%%%%%%%%%%%%%%%%%%%%%%%%%%%%%%%%%%%%%%%%%%%%%%%%%%%%%%%%%%%%%%%%%%%%%%%
\begin{frame}[plain, noframenumbering]
  \begin{beamercolorbox}[wd=\paperwidth, sep=2em]{solarizedcolorbox}
    {\usebeamerfont{title}\usebeamercolor[fg]{title} End EBNF \hfill Start
      Language Theory}
  \end{beamercolorbox}
\end{frame}
%%%%%%%%%%%%%%%%%%%%%%%%%%%%%%%%%%%%%%%%%%%%%%%%%%%%%%%%%%%%%%%%%%%%%%%%%%%%%%%%

%%%%%%%%%%%%%%%%%%%%%%%%%%%%%%%%%%%%%%%%%%%%%%%%%%%%%%%%%%%%%%%%%%%%%%%%%%%%%%%%
\begin{frame}
\frametitle{Deterministic Top-Down Parsing}

We can have a deterministic recursive descent parser if we have no backtracking
\\~\\

To avoid backtracking, we make only correct guesses\\~\\

To make only correct guesses, we restrict the grammar so we can always choose
the correct next step based on the next token
\end{frame}
%%%%%%%%%%%%%%%%%%%%%%%%%%%%%%%%%%%%%%%%%%%%%%%%%%%%%%%%%%%%%%%%%%%%%%%%%%%%%%%%

%%%%%%%%%%%%%%%%%%%%%%%%%%%%%%%%%%%%%%%%%%%%%%%%%%%%%%%%%%%%%%%%%%%%%%%%%%%%%%%%
\begin{frame}
\frametitle{LL(1)}

$LL$ is the class of grammars you've been parsing

\begin{description}
  \item[\alert{L}] for \alert{l}eft-to-right scan of the input
  \item[\alert{L}] for \alert{l}eftmost derivation
\end{description}
~\\
An $LL(k)$ grammar can be parsed top-down with no backtracking looking only at
the next $k$ symbols in the input\\~\\

$LL(1)$ can be parsed with a recursive descent parser with a one token
(or~terminal) look-ahead
\end{frame}
%%%%%%%%%%%%%%%%%%%%%%%%%%%%%%%%%%%%%%%%%%%%%%%%%%%%%%%%%%%%%%%%%%%%%%%%%%%%%%%%

%%%%%%%%%%%%%%%%%%%%%%%%%%%%%%%%%%%%%%%%%%%%%%%%%%%%%%%%%%%%%%%%%%%%%%%%%%%%%%%%
\begin{frame}[plain, noframenumbering]
  \begin{beamercolorbox}[wd=\paperwidth, sep=2em]{solarizedcolorbox}
    {\usebeamerfont{title}\usebeamercolor[fg]{title} End Language Theory}
  \end{beamercolorbox}
\end{frame}
%%%%%%%%%%%%%%%%%%%%%%%%%%%%%%%%%%%%%%%%%%%%%%%%%%%%%%%%%%%%%%%%%%%%%%%%%%%%%%%%
\end{document}
